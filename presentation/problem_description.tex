\documentclass[11pt]{article}

\usepackage[utf8]{inputenc}
\usepackage[T1]{fontenc}
\usepackage{amsthm}
\usepackage{amsmath}
\usepackage{amsfonts}
\usepackage{amssymb}
\usepackage{mathtools}
\usepackage{multicol}
\usepackage{graphicx}
\usepackage{hyperref}
\usepackage[polish]{babel}

\graphicspath{{./images/}}

\newcommand{\set}[1]{\left\lbrace #1 \right\rbrace}
\DeclareMathOperator*{\dist}{dist}
\DeclareMathOperator*{\lin}{lin}

\begin{document}
\section*{Opis problemu}
\subsection*{Śledzenie promieni}
W ramach projektu zaimplementowałem ray tracer powierzchni implicit, działający w czasie rzeczywistym.

\includegraphics[scale=0.7]{promienie.png}

Dla każdego piksela tworzony jest promień zadany równaniem $$x + td$$
dla pewnych wektorów $x$, $d$, gdzie dodatkowo $\|d\| = 1$.
Powierzchnie zadane są równaniem $$ F(x_1, x_2, x_3) = 0$$ gdzie $F$ jest wielomianem stopnia co najwyżej 3. Aby policzyć punkt przecięcia należy rozwiązać równanie $$F(x_1 + td_1, x_2 + td_2, x_3 + td_3) = 0$$ przy czym dla wielomianów stopnia 3 istnieją jawne wzory na pierwiastki. Do obliczenia koloru powierzchni potrzebny jest również wektor normalny, zadany przez gradient $$\nabla F(x_1, x_2, x_3)$$

\subsection*{Źródła światła}
W swoim programie uwzględniłem dwa rodzaje źródeł światła: kierunkowe i punktowe.
\begin{itemize}
    \item \textbf{Źródła kierunkowe} Symulują odległe źródła światła (np. Słońce). Zakładamy, że promienie światłą są równoległe.
    
    \includegraphics[scale=0.7]{kierunkowe.png}

    Dla takich źródeł ilość światła jest wprost proporcjonalna do jego intensywności i kąta padania.

    \item \textbf{Źródła punktowe} Zakładamy, że źródłem światła jest pojedynczy punkt znajdujący się gdzieś na scenie. Intensywność światła spada wraz z kwadratem odległości.
    
    \includegraphics[scale=0.7]{punktowe.png}
\end{itemize}

\subsection*{Cienie}
Czasami dany punkt nie otrzymuje światła, ponieważ jest on zasłonięty przez inny obiekt. Aby to sprawdzić, należy prześledzić dodatkowy promień, emitowany z badanego punktu w kierunku przeciwnym do kierunki padania światła.

\includegraphics[scale=0.7]{cienie.png}

\section*{Implementacja}
Implementacja na GPU tworzy osobny wątek dla każdego rozważanego piksela. Uzyskane jest znaczne przyspieszenie względem implementacji na CPU. Czasy renderowania dla jednej klatki (dla początkowego ustawienia kamery) były następujące:
\begin{itemize}
    \item \texttt{dingdong.yml}: CPU: ok. 800 ms, CUDA: ok. 25 ms (38 FPS)
    \item \texttt{monkey\_saddle.yml}: CPU: ok. 520 ms, CUDA: ok. 17 ms (56 FPS)
\end{itemize}
\section*{Źródło}
\url{https://www.scratchapixel.com}
\end{document}
